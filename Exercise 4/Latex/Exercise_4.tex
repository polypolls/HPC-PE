\documentclass{article}
\usepackage[utf8]{inputenc}
\usepackage{amssymb}
\usepackage{graphicx}
\usepackage{setspace}
\usepackage{listings}
\usepackage{float}
\usepackage{xcolor}
\usepackage{amsmath}
\usepackage{pgfplots}
\usepackage{enumitem}
\usepackage{subcaption}
\usepackage{hyperref}

\title{\textbf{High Performance Computer Architectures Practical Course \\ - Exercise 4 -} \\[10mm]}
\author{Tutorium 1 \\[10mm] David Jordan (6260776) \\[1mm] Florian Rüffer (7454628) \\[1mm] Michael Samjatin (7485765) \\[10mm]}


\lstset{
    language=C++,
    basicstyle=\ttfamily,
    keywordstyle=\color{blue},
    stringstyle=\color{red},
    commentstyle=\color{green},
    numbers=left,
    numberstyle=\normalsize,
    breaklines=true,
    showstringspaces=false,
    frame=single,
    linewidth=1\linewidth,
    captionpos=b
}
\renewcommand{\lstlistingname}{File}% Listing -> Algorithm
\renewcommand{\lstlistlistingname}{List of \lstlistingname s}% List of Listings -> List of Algorithms

\begin{document}
\maketitle
\newpage
\section{Matrix}
\section{Quadratic Equation}
\section{Newton}
\section{Random Access}

\begin{figure}[H]
    \centering
    \includegraphics[scale=0.5]{example-image.png} 
    \caption{Output}
    \label{fig:example}
\end{figure}






\begin{lstlisting}[caption=Matrix.cpp]
TStopwatch timerSIMD;
for( int ii = 0; ii < NIter; ii++ )
    for( int i = 0; i < N; i++ ) {
        for( int j = 0; j < N; j++ ) {
            fvec &aVec = reinterpret_cast<fvec&>(a[i][j]);
            fvec &cVec = reinterpret_cast<fvec&>(c_simd[i][j]);
            cVec = f(aVec);
        }
    }
timerSIMD.Stop();
\end{lstlisting}



\end{document}
