\documentclass{article}
\usepackage[utf8]{inputenc}
\usepackage{amssymb}
\usepackage{graphicx}
\usepackage{setspace}
\usepackage{listings}
\usepackage{float}
\usepackage{xcolor}
\usepackage{amsmath}
\usepackage{pgfplots}
\usepackage{subcaption}
\usepackage{hyperref}

\title{\textbf{High Performance Computer Architectures Practical Course \\ - Exercise 8 -} \\[10mm]}
\author{Tutorium 1 \\[10mm] David Jordan (6260776) \\[1mm] Florian Rüffer (7454628) \\[1mm] Michael Samjatin (7485765) \\[10mm]}


\lstset{
    language=C++,
    basicstyle=\ttfamily,
    keywordstyle=\color{blue},
    stringstyle=\color{red},
    commentstyle=\color{green},
    numbers=left,
    numberstyle=\normalsize,
    breaklines=true,
    showstringspaces=false,
    frame=single,
    linewidth=1\linewidth,
    captionpos=b
}
\renewcommand{\lstlistingname}{File}% Listing -> Algorithm
\renewcommand{\lstlistlistingname}{List of \lstlistingname s}% List of Listings -> List of Algorithms

\begin{document}
\maketitle
\newpage
\section*{1\_First}
\subsection*{Part 1}
\subsection*{Part 2}
\subsection*{Part 3}
In this part we need to SIMDize the program. \\
We start off by creating the vector\_add\_kernel2.cl file,
which contains code to utilize SIMD capabilities.

\begin{lstlisting}[caption=vector\_add\_kernel2.cl]
    __kernel void vector_add(__global float *A, __global float *B, __global float *C) {
    
    
    int i = get_global_id(0);

    
    float4 a = vload4(i, A);
    float4 b = vload4(i, B);

    
    vstore4( a + b, i, C );
}
\end{lstlisting}

\noindent The first line defines a kernel function, which can be executed on an OpenCL device. \\
The function takes three parameters (A, B, C), which can be located in the global memory space.
Furthermore line 4 gets the id is of the current work item and assigns the i-th group of A, B and C to it.
In this case A, B and C are vectors with 4 elements each (line 7,8). \\
Finally line 11 stores the computed values into the $4 \cdot i$ element. \\
\newpage
\noindent Subsequently we will add / load this kernel to our code in the main.cpp file.

\begin{lstlisting}[caption=main.cpp]
#ifdef SIMD
    fp = fopen("vector_add_kernel2.cl", "r");
#else
    fp = fopen("vector_add_kernel.cl", "r");
#endif
    if (!fp) {
        fprintf(stderr, "Failed to load kernel.\n");
        exit(1);
    }
\end{lstlisting}

\noindent This part of the code add the new kernel to the main.cpp code base.
If it is available it will be used, if not another kernel (w/o SIMD) will be used.
If both of the cases above fail, we raise an error. \\\\

\noindent We also should not forget to add the preprocessing macro at the beginning of our code.
This allows us to use SIMD in commands, such as \#ifdef, to switch between scalar or simd execution during compile time.

\begin{lstlisting}[caption=main.cpp]
    #define SIMD
\end{lstlisting}

\noindent Adjust our list size, to be divisible by 4.
\begin{lstlisting}[caption=main.cpp]
    const int LIST_SIZE = 1024;
\end{lstlisting}

\noindent Last but not least, we need to compute the item size.
We do this the same way, in which we have loaded the kernel.

\begin{lstlisting}[caption=main.cpp]
    #ifdef SIMD
    size_t global_item_size = LIST_SIZE/4; // Process the entire lists
#else
    size_t global_item_size = LIST_SIZE;
#endif
    size_t local_item_size = 64; // Process in groups of 64
\end{lstlisting}

\noindent If we have defined SIMD, we will determine the number of work items we need.
For our list size this would result in 256 work items.
If we SIMD is not defined our item size will equal the list size.
Lastly if both of the cases above fail to trigger, we will process in groups of 64.

\subsection*{Part 4}
In this part sub devices are added. \\
Starting off by gathering device information:
\begin{lstlisting}
    cl_platform_id platform_id = NULL;
    cl_device_id device_id = NULL;   
    cl_uint ret_num_devices;
    cl_uint ret_num_platforms;
    cl_int ret = clGetPlatformIDs(1, &platform_id, &ret_num_platforms);
    checkErr(ret, "clGetPlatformIDs");
\end{lstlisting}
The first two lines initialize as NULL and will be used later on to refer the platform and device.
The next two lines hold the number of platforms and devices.
Line 5 denotes an error code (cl\_int), which tells if the platform id was fetched successfully.
Here only one platform is returned and written to platform\_id, due to the first parameter. Also an error check is performed.\\\\

\noindent Here we perform the same task as described above, but for our device.\\
CL\_DEVICE\_TYPE\_CPU restricts us to devices of a certain type.
\begin{lstlisting}
    ret = clGetDeviceIDs( platform_id, CL_DEVICE_TYPE_CPU, 1, 
    &device_id, &ret_num_devices);
    checkErr(ret, "clGetDeviceIDs");
\end{lstlisting}

\noindent These two lines denote variables to hold the number and id of the created subdevices.
\begin{lstlisting}
    cl_uint num_devices_ret;
    cl_device_id  out_devices[80];
\end{lstlisting}

\noindent This part denotes the properties based on which the device will be partitioned into subdevices.
An equal partion means dividing it into as many parts as possible, while each part contains the same number of compute units (in this case 2).
Subsequently the devices are created based on the defined properties.
At the end again an error check is performed.
\begin{lstlisting}
    const cl_device_partition_property props[] = {CL_DEVICE_PARTITION_EQUALLY, 2, 0};
    ret = clCreateSubDevices ( device_id, props, 80 , out_devices , &num_devices_ret );
    checkErr(ret, "clCreateSubDevices");
\end{lstlisting}
\newpage

The other partition methods, include partitioning by counts and affinity domain.
When we partition by count, we can specify the exact number of compute units each subdevice should have.
This allows a more find-grained control.
\begin{lstlisting}
    const cl_device_partition_property props[] = {CL_DEVICE_PARTITION_BY_COUNTS, 1, 1, CL_DEVICE_PARTITION_BY_COUNTS_LIST_END, 0};
\end{lstlisting}
Additionally there is the partitioning method by affinity domain.
Those affinity domains are specific areas of the device (e.g. L1, L2 Cache).
The subdevices will then be created and have an affinity to a certain affinity domain.
The code below will result in an affinity (e.g. faster access) to the L1 Cache.
\begin{lstlisting}
    const cl_device_partition_property props[] = {CL_DEVICE_PARTITION_BY_AFFINITY_DOMAIN, CL_DEVICE_AFFINITY_DOMAIN_L1_CACHE, 0};
\end{lstlisting}
\subsection*{Part 5}
\subsection*{Part 6}
\subsection*{Appendix}

\begin{lstlisting}[caption=example code]
    #pragma omp parallel for num_threads(omp_get_num_procs())
\end{lstlisting}

\begin{figure}[H]
    \centering
    \includegraphics[scale=0.6]{example-image.png} 
    \caption{example image}
    \label{fig:graph}
\end{figure}



\end{document}