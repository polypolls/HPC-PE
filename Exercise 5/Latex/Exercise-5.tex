\documentclass{article}
\usepackage[utf8]{inputenc}
\usepackage{amssymb}
\usepackage{graphicx}
\usepackage{setspace}
\usepackage{listings}
\usepackage{float}
\usepackage{xcolor}
\usepackage{amsmath}
\usepackage{pgfplots}
\usepackage{enumitem}
\usepackage{subcaption}
\usepackage{hyperref}

\title{\textbf{High Performance Computer Architectures Practical Course \\ - Exercise 5 -} \\[10mm]}
\author{Tutorium 1 \\[10mm] David Jordan (6260776) \\[1mm] Florian Rüffer (7454628) \\[1mm] Michael Samjatin (7485765) \\[10mm]}


\lstset{
    language=C++,
    basicstyle=\ttfamily,
    keywordstyle=\color{blue},
    stringstyle=\color{red},
    commentstyle=\color{green},
    numbers=left,
    numberstyle=\normalsize,
    breaklines=true,
    showstringspaces=false,
    frame=single,
    linewidth=1\linewidth,
    captionpos=b
}
\renewcommand{\lstlistingname}{File}% Listing -> Algorithm
\renewcommand{\lstlistlistingname}{List of \lstlistingname s}% List of Listings -> List of Algorithms

\begin{document}
\maketitle
\newpage
\section*{Line 324-430}
First of all the geometry is set up, containing the number of stations / particles, the distance between those stations and the hit distribution.
Those values will result in a linearly spaced distribution of stations along the z-axis.
This simulates a set of tracks with random starting positions and slopes. \\

\noindent Next up, we need to get our tracks fitted, to determine the most probable
trajectory of a particle. For this we need to define the SIMD-ized version of the fitting process or
it will be computed in a scalar way.

\begin{lstlisting}[caption=KFLineFitter.cpp]
  const int NVTracks = NTracks/fvecLen;
  LFTrack<fvec,fvec> vTracks[NVTracks];
  
  CopyTrackHits( tracks, vTracks, NVTracks );
  
    
  LFFitter<fvec,fvec> fit;

  fit.SetSigma( Sigma );
  
#ifdef TIME
  timer.Start(1);
#endif
  for ( int i = 0; i < NVTracks; ++i ) {
    LFTrack<fvec,fvec> &track = vTracks[i];
    fit.Fit( track );
  }
#ifdef TIME
  timer.Stop();
#endif
  
    
  CopyTrackParams( vTracks, tracks, NVTracks );
  
#endif
\end{lstlisting}

\noindent In the first line we calculate the number of vectorized tracks, to assign it to our
list of vectorized tracks in the next line (as the length). In the following steps and lines of code,
we transform our scalar track data into a SIMD vector, fit the tracks parameter to the measure data points
and add Sigma (uncertainty) to our simulation.
The for-loop then iterates over every vectorized track to fit multiple tracks simultaneously.
Last but not least we copy our tracks back to their original format (line 23).
Especially the functions CopyTrackHits and CopyTrackParams are of interest.
This function iterates over the grouped tracks and for those over each hit for every track.
All tracks have the same number of hits, those correspond to the number of stations, therefore
we can treat this parameters as a constant.
For CopyTrackParams we need to iterate over the vectorized tracks and their entries, as well as the
elements of the covariance matrix. \\

\noindent The last lines of code save our results.
If an output file is defined, then the program will output the parameters and covariance matrices for every track.
The output will be stored in the defined file and will look like this:
\begin{lstlisting}[caption=Output]
  0 
  90 -9.13228 -0.0736924
  90 -9.03232 -0.0722956
  0.0862335 0.00135828 3.00843e-05
\end{lstlisting}
Here we will list the line and its corresponding explanation (index number corresponds to line in the output):
\begin{enumerate}
  \item Track index, which increments by one for every track
  \item True values of the parameters (Monte Carlo truth)
  \item Reconstructed / fitted values for the same parameters
  \item Covariance matrix for the fitted parameters
\end{enumerate}
\begin{figure}[H]
    \centering
    \includegraphics[scale=0.3]{example-image.png} 
    \caption{Output}
    \label{fig:example}
  \end{figure}
  







\end{document}