\documentclass{article}
\usepackage[utf8]{inputenc}
\usepackage{amssymb}
\usepackage{graphicx}
\usepackage{setspace}
\usepackage{listings}
\usepackage{float}
\usepackage{xcolor}
\usepackage{amsmath}
\usepackage{pgfplots}
\usepackage{subcaption}
\usepackage{hyperref}

\title{\textbf{High Performance Computer Architectures Practical Course \\ - Exercise 6 -} \\[10mm]}
\author{Tutorium 1 \\[10mm] David Jordan (6260776) \\[1mm] Florian Rüffer (7454628) \\[1mm] Michael Samjatin (7485765) \\[10mm]}


\lstset{
    language=C++,
    basicstyle=\ttfamily,
    keywordstyle=\color{blue},
    stringstyle=\color{red},
    commentstyle=\color{green},
    numbers=left,
    numberstyle=\normalsize,
    breaklines=true,
    showstringspaces=false,
    frame=single,
    linewidth=1\linewidth,
    captionpos=b
}
\renewcommand{\lstlistingname}{File}% Listing -> Algorithm
\renewcommand{\lstlistlistingname}{List of \lstlistingname s}% List of Listings -> List of Algorithms

\begin{document}
\maketitle
\newpage
\section*{Section 1}

First and foremost, we must decide which data should be
grouped and how it should be grouped in order to vectorize
the track fitting procedure.
To achieve maximum independence, M tracks can be handled simultaneously.
The procedure involves:

\section*{FittingDemo\_1}
To accomplish this task we need to adjust the polynomial order of our background function
to the order of three, four and six. We do this with the following code snippets: \\[3mm]

\begin{lstlisting}[caption=Order 3]
    Double_t background(Double_t *x, Double_t *par) {
   return par[0] + par[1]*x[0] + par[2]*x[0]*x[0] + par[3]*x[0]*x[0]*x[0];
}

\end{lstlisting}
\begin{lstlisting}[caption=Order 4]
    Double_t background(Double_t *x, Double_t *par) {
        return par[0] + par[1]*x[0] + par[2]*x[0]*x[0] + par[3]*x[0]*x[0]*x[0] + par[4]*x[0]*x[0]*x[0]*x[0];
     }
     
\end{lstlisting}
\begin{lstlisting}[caption=Order 6]
    Double_t background(Double_t *x, Double_t *par) {
        return par[0] + par[1]*x[0] + par[2]*x[0]*x[0] + par[3]*x[0]*x[0]*x[0] + par[4]*x[0]*x[0]*x[0]*x[0] + par[5]*x[0]*x[0]*x[0]*x[0]*x[0] + par[6]*x[0]*x[0]*x[0]*x[0]*x[0]*x[0];
     }
     

\end{lstlisting}

In the code snippets above the function 'background' takes two parameters x and the value par, which denotes
an array of parameters (six in total).

  
\end{document}